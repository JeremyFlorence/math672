\documentclass{article}
\usepackage{amsfonts, amsmath}
\usepackage[normalem]{ulem}

\newcounter{problem}
\newcounter{solution}

\newcommand\Problem{%
  \stepcounter{problem}%
  \textbf{\theproblem.}~%
  \setcounter{solution}{0}%
}

\newcommand\TheSolution{%
  \textbf{Solution:} %
}

\newcommand\Proof{%
	\textbf{Proof:} %
}

\parindent 0in
\parskip 1em

\begin{document}
Name: Jeremy Florence\\
Course: Math 672\\
Assignment: Homework 9\\
Due: 5/4/17\\

\begin{enumerate}
\item Let $R$ be a commutative domain and let $S=R-\{0\}.$ Prove that $R_{(0)}=R[S^{-1}]$ is a field. This field is called the \emph{field of fractions} of $R$.\\

\Proof Since $R$ is a domain and $0 \not \in S$, we know that $S$ is a multiplicatively closed set. Thus $R_{(0)}$ is a ring. Now we will show that $R_{(0)}$ has multiplicative commutativity. Let $\frac{a}{b},\frac{c}{d} \in R_{(0)}$. Now since $$\frac{a}{b} \cdot \frac{c}{d}=\frac{ac}{bd}$$ and $$\frac{c}{d} \cdot \frac{a}{b}=\frac{ca}{db}$$ we must show that there exists $s \in S$ so that $s(acdb-cabd)=0$. Since $R$ is commutative, we know that $acdb=cabd$. Thus we can choose $s=1$ to satisfy $s(acdb-cabd)=0$, so we now know that $\frac{ac}{bd}=\frac{ca}{db}$. Hence $R_{(0)}$ is commutative under multiplication.

Now we will show that $R_{(0)}$ is closed under multiplicative inverse. Let $\frac{a}{b} \in R_{(0)}$ such that $\frac{a}{b} \neq 0$. Since $\frac{a}{b} \neq 0$, we know that $a \neq 0$. Hence we can choose $\frac{b}{a} \in R_{(0)}$ and we can see that $$\frac{a}{b} \cdot \frac{b}{a}=\frac{ab}{ba}.$$ Now we can choose $1 \in S$ and observe that since $R$ is commutative, $1(ab \cdot 1 - 1 \cdot ba)=0$, so $$\frac{a}{b} \cdot \frac{b}{a}=\frac{ab}{ba}=1.$$ Thus $R_{(0)}$ is closed under multiplicative inverse.

Therefore as $R_{(0)}$ is a ring that is commutative under multiplication and closed under multiplicative inverse, we can conclude that $R_{(0)}$ is a field.

\item As above, let $R \subseteq F$, where $F$ is a (possibly different) field. Prove that the field of fractions of $R$ is isomorphic to the smallest subfield of $F$ which contains $R$.\\

\Proof Observe that for any $b \in R-\{0\}$, $b^{-1} \in F$. Define $f:R_{(0)} \to F$ by $f(\frac{a}{b})=ab^{-1}$ for all $\frac{a}{b} \in R_{(0)}$. We will first show that $f$ is a ring homomorphism. Let $\frac{a}{b},\frac{c}{d} \in R_{(0)}$. Then
\[
	\begin{split}
		f \left(\frac{a}{b} + \frac{c}{d}\right) &= f \left(\frac{ad+bc}{bd} \right)\\
									&=(ad+bc)(d^{-1}b^{-1})\\
									&=add^{-1}b^{-1}+bcd^{-1}b^{-1}\\
									&=ab^{-1}+cd^{-1}\\
									&=f \left( \frac{a}{b} \right) + f \left( \frac{c}{d} \right).
	\end{split}
\]
Also, 
\[
	\begin{split}
		f \left(\frac{a}{b} \cdot \frac{c}{d} \right) &= f \left(\frac{ac}{bd} \right)\\
											&= acd^{-1}b^{-1}\\
											&= ab^{-1} \cdot cd^{-1}\\
											&= f \left( \frac{a}{b} \right) \cdot f \left( \frac{c}{d} \right).
	\end{split}
\]

Therefore we now know that $f$ is a homomorphism.\\

Now we will show that $f$ is injective. Let $\frac{a}{b}, \frac{c}{d} \in R_{(0)}$ so that $f \left( \frac{a}{b} \right) = f \left( \frac{c}{d} \right)$. Then
\[
	\begin{split}
		ab^{-1}&=cd^{-1}\\
		ab^{-1}bd&=cd^{-1}bd\\
				ad&=bc.\\
	\end{split}
\]

Thus $\frac{a}{b}=\frac{c}{d}$ since $ad-bc=0$, so $f$ is injective. Now it is clear that $f:R_{(0)} \to f (R_{(0)})$ is surjective, so $f$ is an isomorphism. Therefore as $f(R_{(0)})$ is a field, and $f(R_{(0)}) \subseteq F$, we now know that $f(R_{(0)})$ is a subfield of $F$. Finally, we can conclude that $R_{(0)}$ is isomorphic to $f(R_{(0)})$, which is the smallest subfield of $F$ which contains $R$.
 
\end{enumerate}


\end{document}