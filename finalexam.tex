\documentclass{article}
\usepackage{amsfonts, amsmath, amssymb}
\usepackage[normalem]{ulem}

\newcounter{problem}
\newcounter{solution}

\newcommand\Problem{%
  \stepcounter{problem}%
  \textbf{\theproblem.}~%
  \setcounter{solution}{0}%
}

\newcommand\TheSolution{%
  \textbf{Solution:} %
}

\newcommand\Proof{%
	\textbf{Proof:} %
}

\parindent 0in
\parskip 1em

\begin{document}
Name: Jeremy Florence \\
Course: Math 672 \\
Assignment: Final Exam \\
Due: 5/19/17

\begin{enumerate}
	\item Show that every quotient ring of a Noetherian ring is a Noetherian Ring. Is the same true for "every subring of..."?
	
	\Proof Let $R$ be a Noetherian ring with an ideal $I$. We  want to show that $R/I$ is also Noetherian. Let $J$ be and ideal of $R/I$. Now by contracting $J$ to $R$ across the canonical map to the quotient $\varphi$, we know that $J^c=\varphi^{-1}(J)$ is an ideal of $R$. Since $R$ is Noetherian, it is finitely generated. That is, there exist $j_1, j_2, ... , j_n$ such that $(j_1, j_2, ..., j_n)=J^c$. Thus it must follow that $(\varphi(j_1),\varphi(j_2),...,\varphi(j_n))=J$. Therefore J is finitely generated and hence Noetherian.
	
	\textbf{Proposition:} Every subring of a Noetherian ring is a Noetherian ring.
	
	\textbf{Counterexample:} Let $k$ be a field. Then the polynomial ring over $k$ with infinitely many variables $R=k[x_1,x_2,...]$ is a commutative domain, but $R$ is not Noetherian because we can keep choosing more variables to extend the chain of ideals: $$(x_1) \subset (x_1, x_2) \subset (x_1, x_2, x_3) \subset ...$$ Now by homework 9, we know that $R$ has a field of fractions $R_{(0)}$. Fields only have two ideals, $(0)$ and $(1)$, so $R_{(0)}$ is clearly Noetherian. We also know that $R \subset R_{(0)}$, so there exists a Noetherian ring which has a subring that is not Noetherian.\\
	
	\item Let $k$ be field. Prove that $k[x,y]$ is \emph{not} a PID.
	
	\Proof For the sake of contradiction, suppose $k[x,y]$ is a PID. Now consider the ideal $(x,y)$. There must exist some $i \in k[x,y]$ so that $(i)=(x,y)$. Hence as $x,y \in (i)$ it must follow that $x$ and $y$ are factors of $i$. However, $x$ and $y$ are both irreducible in $k[x,y]$, so $i$ must be a unit. Therefore $(i)=(x,y)=(1)$, a contradiction. Therefore, $k[x,y]$ is not a PID.\\
	
	\item Find a ring $R$ which has a chain of prime ideals $P_1 \subsetneq P_2 \subsetneq P_3 \subsetneq P_4 \subsetneq R$.
	
	\TheSolution Let $R=\mathbb{Z}[x,y,z]$ and let $p \in \mathbb{Z}$ be prime. Then we have the chain of prime ideals $(p) \subsetneq (p,x) \subsetneq (p,x,y) \subsetneq (p,x,y,z) \subsetneq \mathbb{Z}[x,y,z]$.\\
\\
\\
\\
\\
	
	\item Prove that $\mathbb{Z}[i]$ is a Euclidean domain.
	
	\Proof Define $d: \mathbb{Z}[i]-\{0\} \to \mathbb{N}$ by $d(a+bi)=a^2+b^2$ for all $a+bi \in \mathbb{Z}[i]-\{0\}$. Let $(a+bi),(c+di) \in \mathbb{Z}[i]-\{0\}$. Now since $d$ is the modulus function for Gaussian integers, we know it is multiplicative. Hence
	\[
		\begin{split}
			d((a+bi)(c+di))&=d(a+bi) \cdot d(c+di) \\
							&= (a^2+b^2) \cdot (c^2+d^2).\\
		\end{split}
	\]
	
	Thus as $a,b,c,d \in \mathbb{Z}$, it is clear that $d((a+bi)(c+di)) \geq d(a+bi).$
	
	
	
	\item Let $I \subseteq P \subsetneq R$, where $I$ is an ideal and $P$ is a prime ideal of $R$. Prove that the ring $R_P/I^e$ (the quotient of the localization at $P$ by the extension of the ideal $I$ to that localization) and the ring $(R/I)_{P/I}$ (the localization of the quotient ring $R/I$ at the prime ideal $P/I$) are isomorphic.
	
	\item Let $R$ be a ring and let $f \in R$ be not nilpotent. Let $S=\{f^n:n \in \mathbb{N}\}$. Prove that $R[S^{-1}]$ is isomorphic to $R[z]/(fz-1)$. 
\end{enumerate}


\end{document}