\documentclass{article}
\usepackage{amsfonts, amsmath}
\usepackage[normalem]{ulem}

\newcounter{problem}
\newcounter{solution}

\newcommand\Problem{%
  \stepcounter{problem}%
  \textbf{\theproblem.}~%
  \setcounter{solution}{0}%
}

\newcommand\TheSolution{%
  \textbf{Solution:} %
}

\newcommand\Proof{%
	\textbf{Proof:} %
}

\parindent 0in
\parskip 1em

\begin{document}
Name: Jeremy Florence\\
Course: Math 672\\
Assignment: Homework \#9 problem 3\\
Due: 5/9/17\\

\begin{enumerate}
\item[3.] Classify the units, irreducibles, and zerodivisors of the ring $\mathbb{Z}_{(p)}$. Is this ring a UFD?\\

\TheSolution Note that the multiplicatively closed set for $\mathbb{Z}_{(p)}$ is $S=\mathbb{Z}-\{p\}$.\\ Thus, the units of $\mathbb{Z}_{(p)}$ are the elements in the set $\mathbb{Z}_{(p)}-(\frac{p}{1})$, the set of irreducibles is $\{\frac{a}{b} \in \mathbb{Z}_{(p)} | a=p\}$, and there are no zerodivisors.\\

\textbf{Claim: } $\mathbb{Z}_{(p)}$ is a UFD.\\

\textbf{Proof of Claim: } Recall that in Homework \#8, we proved that $\mathbb{Z}_{(p)}$ has precisely 3 ideals: $(0)$, $(\frac{p}{1})$, and $(1)$. Thus as each of these ideals is principal, $\mathbb{Z}_{(p)}$ is a PID. Therefore $\mathbb{Z}_{(p)}$ is a UFD.
\end{enumerate}


\end{document}